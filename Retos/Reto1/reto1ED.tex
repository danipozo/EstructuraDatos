\documentclass[11pt]{article}
%Gummi|065|=)
\usepackage[utf8]{inputenc}
\usepackage[spanish]{babel}
\usepackage[left=1.5cm,top=2cm,right=1.5cm,bottom=2.5cm]{geometry} 
\usepackage{ upgreek }
\usepackage{ amssymb }
\usepackage{amsmath}
\usepackage{helvet}

\usepackage{minted}
\renewcommand{\familydefault}{\sfdefault}
\title{\textbf{DOCUMENTO DE TRABAJO!}}
\author{JaSaMaldo}
\date{}

\begin{document}

\maketitle

\begin{enumerate}
\item  Usando la notación O
, determinar la eficiencia de los siguientes segmentos de
código

Analizaremos ambos códigos escribiendo en comentarios la eficiencia de cada parte

\subsection*{Código 1}

\begin{minted}[escapeinside=||,mathescape=true]{c++}

	int n,j; //O(2)
	int i = 1; int x = 0; //O(4)
	do{
		j = 1; //O(1)
		while( j <= n){ // 0(1) 
			j = j*2; // O(2)
			x++; //O(1)
		} // El bucle es de eficiencia O($l_2(n)$) * O(1)
		i++; //O(1)
		
	}while(i<=n); //O(n-1)
	
\end{minted}

Analicemos porque la eficiencia del bucle interior es $l_2(n)$:

Llamemos $f(n)$ al número de veces que se ejecuta el bucle dependiendo del valor n,  entonces tenemos (por j multiplicarse por 2) que:

\[ f(x) = f\left(\frac{x}{2}\right) + 1 = f\left(\frac{x}{4}\right) + 1 + 1 = ... = \underbrace{1 + 1 + ... + 1}_{log_2 (n)} \]

Así, la eficiencia de nuestro código es:

\[
O(2) + O(4) + O(n-1)*(O(l_2n)*(O(1)+O(2)+O(1))+O(1)) = 
\]
\[
O(6) + O(n-1)*(O(4l_2n + 1)) = 
\]
\[
O(6) + O(4nl_2n + n - 4l_2n - 1 ) = 
\]
\[
O((4n - 4)l_2n + n + 5)
\]

Que, como sabemos por la notación $O$, podemos reducir todo eso en el que tenga mayor relevancia, quedando así como resultado final: 
\[O(nl_2n)\]
\subsection*{Código 2}

\begin{minted}[escapeinside=||,mathescape=true]{c++}
	
	int n,j; int i = 2; int x = 0; //O(6)
	do{
		j = 1; // O(1)
		while(j<=i){ // O(1)
			j = j*2; //O(2) 
			x++; // O(1)
		} // El bucle tiene una eficiencia de  O($log_2(i)$) * O(1)
		i++; // O(1)
	}while(i<=n) //O(n-2)
\end{minted}

Analicemos un poco cuantas iteraciones realizan nuestros bucles anidados con respecto de n:
El bucle exterior se ejecuta $n-2$ veces,  donde, en cada una de ellas, el bucle interior realiza $l_2(i)$ donde $i$ va creciendo en cada iteración del bucle exterior desde 2 hasta n. Es decir, tenemos:
\[ \sum_{i=2}^{n} log_2  (i) = log_2 (\prod_{i=2}^{n}i) = log_2 (n!) \]

Omitiendo ya, aquellas iteraciones que sabemos que pasarian a ser nulas por la notación O grande.

	
\end{enumerate}

\end{document}