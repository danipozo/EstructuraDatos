\documentclass[11pt]{article}
%Gummi|065|=)
\usepackage[utf8]{inputenc}
\usepackage[spanish]{babel}
\usepackage[left=1.5cm,top=2cm,right=1.5cm,bottom=2.5cm]{geometry} 
\usepackage{ upgreek }
\usepackage{ amssymb }
\usepackage{amsmath}
\usepackage{helvet}

\usepackage{minted}
\renewcommand{\familydefault}{\sfdefault}
\title{\textbf{Reto 1 - Estructura de Datos}}
\author{Javier Sáez Maldonado\\
Luis Ortega Andrés}
\date{}

\begin{document}

\maketitle

\begin{enumerate}
\item  Usando la notación O
, determinar la eficiencia de los siguientes segmentos de
código

Analizaremos ambos códigos escribiendo en comentarios la eficiencia de cada parte

\subsection*{Código 1}

\begin{minted}[escapeinside=||,mathescape=true]{c++}

	int n,j; //O(2)
	int i = 1; int x = 0; //O(4)
	do{
		j = 1; //O(1)
		while( j <= n){ // 0(1) 
			j = j*2; // O(2)
			x++; //O(1)
		} // El bucle es de eficiencia O($l_2(n)$) * O(1)
		i++; //O(1)
		
	}while(i<=n); //O(n-1)
	
\end{minted}

Analicemos porque la eficiencia del bucle interior es $l_2(n)$:

Llamemos $f(n)$ al número de veces que se ejecuta el bucle dependiendo del valor n,  entonces tenemos (por j multiplicarse por 2) que:

\[ f(x) = f\left(\frac{x}{2}\right) + 1 = f\left(\frac{x}{4}\right) + 1 + 1 = ... = \underbrace{1 + 1 + ... + 1}_{log_2 (n)} \]

Así, la eficiencia de nuestro código es:

\[
O(2) + O(4) + O(n-1)*(O(l_2n)*(O(1)+O(2)+O(1))+O(1)) = 
\]
\[
O(6) + O(n-1)*(O(4l_2n + 1)) = 
\]
\[
O(6) + O(4nl_2n + n - 4l_2n - 1 ) = 
\]
\[
O((4n - 4)l_2n + n + 5)
\]

Que, como sabemos por la notación $O$, podemos reducir todo eso en el que tenga mayor relevancia, quedando así como resultado final: 
\[O(nl_2n)\]
\subsection*{Código 2}

\begin{minted}[escapeinside=||,mathescape=true]{c++}
	
	int n,j; int i = 2; int x = 0; //O(6)
	do{
		j = 1; // O(1)
		while(j<=i){ // O(1)
			j = j*2; //O(2) 
			x++; // O(1)
		} // El bucle tiene una eficiencia de  O($log_2(i)$) * O(1)
		i++; // O(1)
	}while(i<=n) //O(n-2)
\end{minted}

Analicemos un poco cuantas iteraciones realizan nuestros bucles anidados con respecto de n:
El bucle exterior se ejecuta $n-2$ veces,  donde, en cada una de ellas, el bucle interior realiza $l_2(i)$ donde $i$ va creciendo en cada iteración del bucle exterior desde 2 hasta n. Es decir, tenemos:
\[ \sum_{i=2}^{n} log_2  (i) = log_2 (\prod_{i=2}^{n}i) = log_2 (n!) \]

Omitiendo ya, aquellas iteraciones que sabemos que pasarian a ser nulas por la notación O grande.

	
\end{enumerate}


\section*{Ejercicio 2}

Para cada función $f(n)$ y cada tiempo $t$ de la tabla siguiente, determinar el mayor tamaño de un problema que puede ser resuelto en un tiempo $t$ (suponiendo que el algoritmo para resolver el problema tarda $f(n)$ microsegundos, es decir $f(n)*10^{-6}$ sg.)\\

Para resolver este problema, lo hacemos de la siguiente forma:

\[
10^{-6}f(n) = t \Rightarrow n = f^{-1}(t*10^6)
\]

El resultado es el siguiente.


\begin{tabular}{|l|l|llll}
\cline{1-2}
$f(n)$  & t             &                                       &                                              &                                                  &                                                        \\ \hline
        & 1sg           & \multicolumn{1}{l|}{1h}               & \multicolumn{1}{l|}{1semana}                 & \multicolumn{1}{l|}{1 año}                       & \multicolumn{1}{l|}{1000 años}                         \\ \hline
$l_2n$  & $10^{300000}$ & \multicolumn{1}{l|}{$2^{10^6 *3600}$} & \multicolumn{1}{l|}{$2^{10^6 *3600 * 24*7}$} & \multicolumn{1}{l|}{$2^{10^6*3600 * 24*7 * 52}$} & \multicolumn{1}{l|}{$2^{10^6 *3600 * 24*7 * 52*1000}$} \\ \hline
$n$     & 100160256     & \multicolumn{1}{l|}{360576923076,9}   & \multicolumn{1}{l|}{$6,057692308*10^{13}$}   & \multicolumn{1}{l|}{$3.15*10^{15}$}              & \multicolumn{1}{l|}{$3.15*10^{18}$}                    \\ \hline
$nl_2n$ & 62746,1       & \multicolumn{1}{l|}{$1,33*10^8$}      & \multicolumn{1}{l|}{$1,77*10^{10}$}          & \multicolumn{1}{l|}{$7,9*10^{11}$}               & \multicolumn{1}{l|}{$6.39*10^{14}$}                    \\ \hline
$n^3$   & 100           & \multicolumn{1}{l|}{1532,6}           & \multicolumn{1}{l|}{8456,8}                  & \multicolumn{1}{l|}{31565}                       & \multicolumn{1}{l|}{315649}                            \\ \hline
$2^n$   & 19,93         & \multicolumn{1}{l|}{31,7}             & \multicolumn{1}{l|}{39,13}                   & \multicolumn{1}{l|}{44,8}                        & \multicolumn{1}{l|}{54,8}                              \\ \hline
$n!$    & 9,44          & \multicolumn{1}{l|}{12,78}            & \multicolumn{1}{l|}{14,6}                    & \multicolumn{1}{l|}{16}                          & \multicolumn{1}{l|}{18,2}                              \\ \hline
\end{tabular}
\end{document}