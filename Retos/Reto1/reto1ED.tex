\documentclass[11pt]{article}
%Gummi|065|=)
\usepackage[utf8]{inputenc}
\usepackage[spanish]{babel}
\usepackage[left=1.5cm,top=2cm,right=1.5cm,bottom=2.5cm]{geometry} 
\usepackage{ upgreek }
\usepackage{ amssymb }
\usepackage{helvet}

\usepackage{minted}
\renewcommand{\familydefault}{\sfdefault}
\title{\textbf{Plantilla}}
\author{Javier Sáez Maldonado}
\date{}

\begin{document}

\maketitle

\begin{enumerate}
\item  Usando la notación O
, determinar la eficiencia de los siguientes segmentos de
código

Analizaremos ambos códigos escribiendo en comentarios la eficiencia de cada parte

\subsection*{Código 1}

\begin{minted}{c++}

	int n,j; //O(2)
	int i = 1; int x = 0; //O(4)
	do{
		j = 1; //O(1)
		while( j <= n)/* O(1) */{ 
			j = j*2; // O(2)
			x++; //O(1)
		} // El bucle es de eficiencia O(l2(n)) * O(1) 
		i++; //O(1)
		
	}while(i<=n); //O(n-1)
	
\end{minted}


Así, la eficiencia de nuestro código es:

\[
O(2) + O(4) + O(n-1)*(O(l_2n)*(O(1)+O(2)+O(1))+O(1)) = 
\]
\[
O(6) + O(n-1)*(O(4l_2n + 1)) = 
\]
\[
O(6) + O(4nl_2n + n - 4l_2n - 1 ) = 
\]
\[
O((4n - 4)l_2n + n + 5)
\]

Que, como sabemos por la notación $O$, podemos reducir todo eso en el que tenga mayor relevancia, quedando así como resultado final: 
\[O(nl_2n)\]
\subsection*{Código 2}

\begin{minted}{c++}
	
	int n,j; int i = 2; int x = 0;
	do{
		j = 1
		while(j<=i){
		j = j*2; //O(2) 
		x++;
		
	}while(i<=n)
\end{minted}

	
\end{enumerate}

\end{document}